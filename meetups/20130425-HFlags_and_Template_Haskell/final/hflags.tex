\documentclass[]{article}
\usepackage[T1]{fontenc}
\usepackage{lmodern}
\usepackage{amssymb,amsmath}
\usepackage[a4paper]{geometry}
\usepackage{ifxetex,ifluatex}
\usepackage{fixltx2e} % provides \textsubscript
% use upquote if available, for straight quotes in verbatim environments
\IfFileExists{upquote.sty}{\usepackage{upquote}}{}
\ifnum 0\ifxetex 1\fi\ifluatex 1\fi=0 % if pdftex
  \usepackage[utf8]{inputenc}
\else % if luatex or xelatex
  \usepackage{fontspec}
  \ifxetex
    \usepackage{xltxtra,xunicode}
  \fi
  \defaultfontfeatures{Mapping=tex-text,Scale=MatchLowercase}
  \newcommand{\euro}{€}
\fi
% use microtype if available
\IfFileExists{microtype.sty}{\usepackage{microtype}}{}
\usepackage{color}
\usepackage{fancyvrb}
\usepackage{soul}
\DefineShortVerb[commandchars=\\\{\}]{\|}
\DefineVerbatimEnvironment{Highlighting}{Verbatim}{commandchars=\\\{\}}
% Add ',fontsize=\small' for more characters per line
\newenvironment{Shaded}{}{}
\newcommand{\KeywordTok}[1]{\textcolor[rgb]{0.00,0.44,0.13}{\textbf{{#1}}}}
\newcommand{\DataTypeTok}[1]{\textcolor[rgb]{0.56,0.13,0.00}{{#1}}}
\newcommand{\DecValTok}[1]{\textcolor[rgb]{0.25,0.63,0.44}{{#1}}}
\newcommand{\BaseNTok}[1]{\textcolor[rgb]{0.25,0.63,0.44}{{#1}}}
\newcommand{\FloatTok}[1]{\textcolor[rgb]{0.25,0.63,0.44}{{#1}}}
\newcommand{\CharTok}[1]{\textcolor[rgb]{0.25,0.44,0.63}{{#1}}}
\newcommand{\StringTok}[1]{\textcolor[rgb]{0.25,0.44,0.63}{{#1}}}
\newcommand{\CommentTok}[1]{\textcolor[rgb]{0.38,0.63,0.69}{\textit{{#1}}}}
\newcommand{\OtherTok}[1]{\textcolor[rgb]{0.00,0.44,0.13}{{#1}}}
\newcommand{\AlertTok}[1]{\textcolor[rgb]{1.00,0.00,0.00}{\textbf{{#1}}}}
\newcommand{\FunctionTok}[1]{\textcolor[rgb]{0.02,0.16,0.49}{{#1}}}
\newcommand{\RegionMarkerTok}[1]{{#1}}
\newcommand{\ErrorTok}[1]{\textcolor[rgb]{1.00,0.00,0.00}{\textbf{{#1}}}}
\newcommand{\NormalTok}[1]{{#1}}
\ifxetex
  \usepackage[setpagesize=false, % page size defined by xetex
              unicode=false, % unicode breaks when used with xetex
              xetex]{hyperref}
\else
  \usepackage[unicode=true]{hyperref}
\fi
\hypersetup{breaklinks=true,
            bookmarks=true,
            pdfauthor={Gergely Risko (gergely@risko.hu, errge@google.com)},
            pdftitle={Cmdline libraries, intro to TemplateHaskell},
            colorlinks=true,
            urlcolor=blue,
            linkcolor=magenta,
            pdfborder={0 0 0}}
\urlstyle{same}  % don't use monospace font for urls
\setlength{\parindent}{0pt}
\setlength{\parskip}{6pt plus 2pt minus 1pt}
\setlength{\emergencystretch}{3em}  % prevent overfull lines
\setcounter{secnumdepth}{0}

\title{Cmdline libraries, intro to TemplateHaskell}
\author{Gergely Risko (\href{mailto:gergely@risko.hu}{gergely@risko.hu},
                {\st{errge@google.com}})}
\date{HaskellerZ meetup - April 25th, 2013}

\begin{document}
\maketitle

\section{Agenda 1}

Demonstration of some command-line option parsing libraries:

\begin{itemize}
\item
  cmdtheline
\item
  getopt
\item
  HFlags

\begin{Shaded}
\begin{Highlighting}[]
\OtherTok{\{-# LANGUAGE TemplateHaskell #-\}}

\KeywordTok{import} \NormalTok{HFlags}

\NormalTok{defineFlag }\StringTok{"name"} \StringTok{"Indiana Jones"} \StringTok{"Who to greet"}
\NormalTok{defineFlag }\StringTok{"repeat"} \NormalTok{(}\DecValTok{3} \FunctionTok{+} \DecValTok{4}\OtherTok{ ::} \DataTypeTok{Int}\NormalTok{) }\StringTok{"Number of repeats"}

\NormalTok{main }\FunctionTok{=} \KeywordTok{do} \NormalTok{s }\OtherTok{<-} \FunctionTok{$}\NormalTok{(initHFlags }\StringTok{"Simple program v0.1"}\NormalTok{)}
          \FunctionTok{sequence_} \FunctionTok{$} \FunctionTok{replicate} \NormalTok{flags_repeat }\FunctionTok{$} \FunctionTok{putStrLn} \NormalTok{flags_name}
\end{Highlighting}
\end{Shaded}

  That's all, after that you have \texttt{-{}-version},
  \texttt{-{}-help} and \texttt{-{}-undefok} for free.
\end{itemize}

\section{Agenda 2}

I'll also show some TemplateHaskell, a macro language to reduce
boilerplate.

\begin{Shaded}
\begin{Highlighting}[]
\OtherTok{\{-# LANGUAGE TemplateHaskell #-\}}

\KeywordTok{module} \DataTypeTok{Tup} \KeywordTok{where}

\KeywordTok{import} \NormalTok{Language.Haskell.TH}

\NormalTok{get n k }\FunctionTok{=} \KeywordTok{do}
  \NormalTok{name }\OtherTok{<-} \NormalTok{newName }\StringTok{"wanted"}
  \KeywordTok{let} \NormalTok{arg }\FunctionTok{=} \NormalTok{tupP (}\FunctionTok{replicate} \NormalTok{k wildP }\FunctionTok{++}
                  \NormalTok{[varP name] }\FunctionTok{++} \FunctionTok{replicate} \NormalTok{(n }\FunctionTok{-} \NormalTok{k }\FunctionTok{-} \DecValTok{1}\NormalTok{) wildP)}
  \KeywordTok{let} \NormalTok{body }\FunctionTok{=} \NormalTok{varE name}
  \NormalTok{lamE [arg] body}
\end{Highlighting}
\end{Shaded}

\begin{Shaded}
\begin{Highlighting}[]
\OtherTok{\{-# LANGUAGE TemplateHaskell #-\}}

\KeywordTok{import} \NormalTok{Tup}

\NormalTok{main }\FunctionTok{=} \FunctionTok{print} \NormalTok{(}\FunctionTok{$}\NormalTok{(get }\DecValTok{6} \DecValTok{3}\NormalTok{) (}\DecValTok{0}\NormalTok{,}\DecValTok{1}\NormalTok{,}\DecValTok{2}\NormalTok{,}\DecValTok{3}\NormalTok{,}\DecValTok{4}\NormalTok{,}\DecValTok{5}\NormalTok{))}
\end{Highlighting}
\end{Shaded}

\section{cmdtheline}

CmdTheLine uses applicative functors to provide a declarative,
compositional mechanism for defining command-line programs by lifting
regular Haskell functions over argument parsers. What's the problem?

\begin{Shaded}
\begin{Highlighting}[]
\KeywordTok{import} \NormalTok{Control.Applicative}
\KeywordTok{import} \NormalTok{System.Console.CmdTheLine}

\OtherTok{name ::} \DataTypeTok{Term} \DataTypeTok{String}
\NormalTok{name }\FunctionTok{=} \NormalTok{value }\FunctionTok{$} \NormalTok{opt }\StringTok{"Indiana Jones"} \FunctionTok{$} \NormalTok{optInfo [ }\StringTok{"name"}\NormalTok{, }\StringTok{"n"} \NormalTok{]}

\OtherTok{times ::} \DataTypeTok{Term} \DataTypeTok{Int}
\NormalTok{times }\FunctionTok{=} \NormalTok{value }\FunctionTok{$} \NormalTok{opt }\DecValTok{4} \FunctionTok{$} \NormalTok{optInfo [ }\StringTok{"times"}\NormalTok{, }\StringTok{"t"} \NormalTok{]}

\NormalTok{hello name times }\FunctionTok{=} \FunctionTok{sequence_} \FunctionTok{$} \FunctionTok{replicate} \NormalTok{times }\FunctionTok{$} \FunctionTok{putStrLn} \NormalTok{name}

\NormalTok{term }\FunctionTok{=} \NormalTok{hello }\FunctionTok{<$>} \NormalTok{name }\FunctionTok{<*>} \NormalTok{times}

\NormalTok{termInfo }\FunctionTok{=} \NormalTok{defTI \{ termName }\FunctionTok{=} \StringTok{"Hello"}\NormalTok{, version }\FunctionTok{=} \StringTok{"1.0"} \NormalTok{\}}

\NormalTok{main }\FunctionTok{=} \NormalTok{run ( term, termInfo )}
\end{Highlighting}
\end{Shaded}

\section{Monad -\textgreater{} Applicative -\textgreater{} Functor}

Every Monad is an Applicative and every Applicative is a Functor, here
is why:

\begin{Shaded}
\begin{Highlighting}[]
\FunctionTok{fmap}\OtherTok{   ::} \KeywordTok{Functor} \NormalTok{f     }\OtherTok{=>} \NormalTok{(a }\OtherTok{->} \NormalTok{b) }\OtherTok{->} \NormalTok{f a }\OtherTok{->} \NormalTok{f b}
\OtherTok{(>>=)  ::} \KeywordTok{Monad} \NormalTok{m       }\OtherTok{=>} \NormalTok{m a }\OtherTok{->} \NormalTok{(a }\OtherTok{->} \NormalTok{m b) }\OtherTok{->} \NormalTok{m b}
\OtherTok{(<*>)  ::} \KeywordTok{Applicative} \NormalTok{f }\OtherTok{=>} \NormalTok{f (a }\OtherTok{->} \NormalTok{b) }\OtherTok{->} \NormalTok{f a }\OtherTok{->} \NormalTok{f b}
\FunctionTok{return}\OtherTok{ ::} \KeywordTok{Monad} \NormalTok{m       }\OtherTok{=>} \NormalTok{a }\OtherTok{->} \NormalTok{m a}
\OtherTok{pure   ::} \KeywordTok{Applicative} \NormalTok{f }\OtherTok{=>} \NormalTok{a }\OtherTok{->} \NormalTok{f a}

\OtherTok{fmapFromAp ::} \KeywordTok{Applicative} \NormalTok{f }\OtherTok{=>} \NormalTok{(a }\OtherTok{->} \NormalTok{b) }\OtherTok{->} \NormalTok{f a }\OtherTok{->} \NormalTok{f b}
\NormalTok{fmapFromAp pf av }\FunctionTok{=} \NormalTok{(pure pf) }\FunctionTok{<*>} \NormalTok{av}

\OtherTok{fmapFromM ::} \KeywordTok{Monad} \NormalTok{m }\OtherTok{=>} \NormalTok{(a }\OtherTok{->} \NormalTok{b) }\OtherTok{->} \NormalTok{m a }\OtherTok{->} \NormalTok{m b}
\NormalTok{fmapFromM pf mv }\FunctionTok{=} \NormalTok{mv }\FunctionTok{>>=} \NormalTok{(\textbackslash{}x }\OtherTok{->} \FunctionTok{return} \FunctionTok{$} \NormalTok{pf x)}

\OtherTok{apFromM ::} \KeywordTok{Monad} \NormalTok{m }\OtherTok{=>} \NormalTok{m (a }\OtherTok{->} \NormalTok{b) }\OtherTok{->} \NormalTok{m a }\OtherTok{->} \NormalTok{m b}
\NormalTok{apFromM mf mv }\FunctionTok{=} \NormalTok{mv }\FunctionTok{>>=} \NormalTok{pvToRet}
  \KeywordTok{where}
    \NormalTok{pvToRet pv }\FunctionTok{=} \NormalTok{mf }\FunctionTok{>>=} \NormalTok{pvpfToRet pv}
    \NormalTok{pvpfToRet pv pf }\FunctionTok{=} \FunctionTok{return} \NormalTok{(pf pv)}

\CommentTok{-- of course, fmapFromM was redundant}
\NormalTok{fmapFromM2 pf mv }\FunctionTok{=} \NormalTok{(}\FunctionTok{return} \NormalTok{pf) }\OtherTok{`apFromM`} \NormalTok{mv}
\end{Highlighting}
\end{Shaded}

So, Monads are more specific than Applicatives, and in turn Aplicatives
are more specific than Functors.

\section{Inside cmdtheline: why Term is an Applicative?}

The important data structure inside cmdtheline is the Term.

Why is that an Applicative? Why not only Fuctor? Why not Monad if
already an Applicative?

\begin{Shaded}
\begin{Highlighting}[]
\KeywordTok{data} \DataTypeTok{Term} \NormalTok{a }\FunctionTok{=} \DataTypeTok{Term} \NormalTok{[}\DataTypeTok{ArgInfo}\NormalTok{] (}\DataTypeTok{Yield} \NormalTok{a)}
\KeywordTok{type} \DataTypeTok{Yield} \NormalTok{a }\FunctionTok{=} \DataTypeTok{EvalInfo} \OtherTok{->} \DataTypeTok{CmdLine} \OtherTok{->} \DataTypeTok{Err} \NormalTok{a}

\KeywordTok{instance} \KeywordTok{Functor} \DataTypeTok{Term} \KeywordTok{where}
  \FunctionTok{fmap} \FunctionTok{=} \NormalTok{yield }\FunctionTok{.} \NormalTok{result }\FunctionTok{.} \NormalTok{result }\FunctionTok{.} \FunctionTok{fmap}
    \KeywordTok{where}
    \NormalTok{yield f (}\DataTypeTok{Term} \NormalTok{ais y) }\FunctionTok{=} \DataTypeTok{Term} \NormalTok{ais (f y)}
    \NormalTok{result }\FunctionTok{=} \NormalTok{(}\FunctionTok{.}\NormalTok{)}

\KeywordTok{instance} \KeywordTok{Applicative} \DataTypeTok{Term} \KeywordTok{where}
  \NormalTok{pure v }\FunctionTok{=} \DataTypeTok{Term} \NormalTok{[] (\textbackslash{} _ _ }\OtherTok{->} \FunctionTok{return} \NormalTok{v)}

  \NormalTok{(}\DataTypeTok{Term} \NormalTok{args f) }\FunctionTok{<*>} \NormalTok{(}\DataTypeTok{Term} \NormalTok{args' v) }\FunctionTok{=} \DataTypeTok{Term} \NormalTok{(args }\FunctionTok{++} \NormalTok{args') wrapped}
    \KeywordTok{where}
    \NormalTok{wrapped ei cl }\FunctionTok{=} \NormalTok{f ei cl }\FunctionTok{<*>} \NormalTok{v ei cl}
\end{Highlighting}
\end{Shaded}

It has to be an Applicative, so we can merge different parts of a
program that uses different arguments to a bigger program that uses all
of the arguments. Functor is not enough, that only ever cares about one
context.

But it can't be a Monad, because Monads are too specific to be able to
evaluate a context before the computation. So we won't be able to (++)
together all the args as we do in the Applicative instance.

\section{Cmdtheline summary}

Pros:

\begin{itemize}
\item
  Very clever idea
\item
  Clean and readable implementation
\item
  Good documentation with lot of examples
\item
  Lot of features: e.g.~man page generation
\end{itemize}

Cons:

\begin{itemize}
\item
  The approach needed in Main.hs is a bit cryptic for a beginner
\item
  Tutorial website is down and not included on GitHub
\item
  Composition with options in libraries is cumbersome
\end{itemize}

\section{GetOpt}

Similar to C getopt, included in base.

\begin{Shaded}
\begin{Highlighting}[]
\KeywordTok{import} \NormalTok{System.Console.GetOpt}
\KeywordTok{import} \NormalTok{System.Environment}

\KeywordTok{data} \DataTypeTok{MyOptions} \FunctionTok{=}
  \DataTypeTok{MyOptions} \NormalTok{\{}\OtherTok{ name ::} \DataTypeTok{String}
            \NormalTok{,}\OtherTok{ times ::} \DataTypeTok{Int} \NormalTok{\}}

\NormalTok{def }\FunctionTok{=} \DataTypeTok{MyOptions} \NormalTok{\{ name }\FunctionTok{=} \StringTok{"Indiana Jones"}\NormalTok{, times }\FunctionTok{=} \DecValTok{4} \NormalTok{\}}

\NormalTok{options }\FunctionTok{=}
  \NormalTok{[ }\DataTypeTok{Option} \NormalTok{[}\CharTok{'n'}\NormalTok{] [}\StringTok{"name"}\NormalTok{]}
    \NormalTok{(}\DataTypeTok{ReqArg} \NormalTok{(\textbackslash{}name r }\OtherTok{->} \NormalTok{r \{ name }\FunctionTok{=} \NormalTok{name \}) }\StringTok{"NAME"}\NormalTok{) }\StringTok{"Who to greet"}
  \NormalTok{, }\DataTypeTok{Option} \NormalTok{[}\CharTok{'t'}\NormalTok{] [}\StringTok{"times"}\NormalTok{]}
    \NormalTok{(}\DataTypeTok{ReqArg} \NormalTok{(\textbackslash{}times r }\OtherTok{->} \NormalTok{r \{ times }\FunctionTok{=} \FunctionTok{read} \NormalTok{times\}) }\StringTok{"INT"}\NormalTok{) }\StringTok{"How many times"} \NormalTok{]}

\NormalTok{run opts }\FunctionTok{=} \FunctionTok{sequence_} \FunctionTok{$} \FunctionTok{replicate} \NormalTok{(times opts) }\FunctionTok{$} \FunctionTok{putStrLn} \NormalTok{(name opts)}

\NormalTok{main }\FunctionTok{=} \KeywordTok{do}
  \NormalTok{argv }\OtherTok{<-} \NormalTok{getArgs}
  \KeywordTok{case} \NormalTok{getOpt }\DataTypeTok{Permute} \NormalTok{options argv }\KeywordTok{of}
    \NormalTok{(o,n,[]  ) }\OtherTok{->} \NormalTok{run (}\FunctionTok{foldr} \NormalTok{(}\FunctionTok{$}\NormalTok{) def o)}
    \NormalTok{(_,_,errs) }\OtherTok{->} \FunctionTok{ioError} \NormalTok{(}\FunctionTok{userError} \FunctionTok{$} \FunctionTok{concat} \NormalTok{errs }\FunctionTok{++} \NormalTok{usageInfo }\StringTok{"example"} \NormalTok{options)}
\end{Highlighting}
\end{Shaded}

\section{GetOpt summary}

Pros:

\begin{itemize}
\item
  Simple, by the book solution
\item
  Clean control flow
\item
  Included in base (so comes with GHC, you don't even need the platform)
\end{itemize}

Cons:

\begin{itemize}
\item
  Even the simple thing is long
\item
  Had to roll our own foldr (\$) def to handle default args and
  unordered opts
\item
  Composition with options in libraries gonna be a nightmare
\end{itemize}

HFlags uses GetOpt for parsing.

\section{HFlags}

\begin{Shaded}
\begin{Highlighting}[]
\OtherTok{\{-# LANGUAGE TemplateHaskell #-\}}

\KeywordTok{import} \NormalTok{HFlags}

\NormalTok{defineFlag }\StringTok{"name"} \StringTok{"Indiana Jones"} \StringTok{"Who to greet"}
\NormalTok{defineFlag }\StringTok{"repeat"} \NormalTok{(}\DecValTok{3} \FunctionTok{+} \DecValTok{4}\OtherTok{ ::} \DataTypeTok{Int}\NormalTok{) }\StringTok{"Number of repeats"}

\NormalTok{main }\FunctionTok{=} \KeywordTok{do} \NormalTok{s }\OtherTok{<-} \FunctionTok{$}\NormalTok{(initHFlags }\StringTok{"Simple program v0.1"}\NormalTok{)}
          \FunctionTok{sequence_} \FunctionTok{$} \FunctionTok{replicate} \NormalTok{flags_repeat }\FunctionTok{$} \FunctionTok{putStrLn} \NormalTok{flags_name}
\end{Highlighting}
\end{Shaded}

Pros:

\begin{itemize}
\item
  Simple definition of flags and pure access to them
\item
  Very easy and nice API free of boilerplate
\item
  Support for \texttt{-{}-undefok} (needed for complex deployments!),
  \texttt{-{}-help}
\item
  Libraries can introduce flags on their own, no things to pass around
\end{itemize}

Cons:

\begin{itemize}
\item
  Impure, people will yell at me
\item
  Uses TemplateHaskell to have the nice interface (portability issues)
\item
  main has to splice in the initialization function with TH
\end{itemize}

\section{Demo time!}

\begin{itemize}
\item
  SimpleExample
\item
  ImportExample with collision detection
\item
  Package example with cabal
\item
  Complex example
\end{itemize}

\section{Template Haskell Demo Environment}

Template Haskell is easy experiment with inside GHCi, ideal for learning
and tinkering.

\begin{itemize}
\item
  :set -XTemplateHaskell
\item
  useful: pprintQ q = putStrLn . pprint =\textless{}\textless{} runQ q
\item
  GHC flag for watching splices getting built: :set -ddump-splices
\item
  work through intro.ihs
\item
  work through reify.ihs
\item
  work through HFlags.hs
\end{itemize}

\section{Thank you}

Questions, remarks?

Template Haskell in the haskellwiki:
\url{http://www.haskell.org/haskellwiki/Template_Haskell}

Paper on TemplateHaskell:
\url{http://research.microsoft.com/~simonpj/papers/meta-haskell/}

Design notes for V2:
\url{http://research.microsoft.com/en-us/um/people/simonpj/tmp/notes2.ps}

Reference docs on Hackage:
\url{http://hackage.haskell.org/package/template-haskell}

HFlags on Hackage: \url{http://hackage.haskell.org/package/hflags}

\end{document}
